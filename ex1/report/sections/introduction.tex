\section{Introduction}

We were given the task of making ``a program which enables a player to
control a `paddle' on the row of LED-diodes'', as stated in the
compendium given in the course Microcontroller System Design (TDT4258).
The main goal of this task was to get a good understanding in how to
program I/O and interrupt handling in assembly code for the AVR32
microprocessor, as well as understand and learn how to use the
GNU-toolchain that is used to assemble, link and debug this
microprocessor.

To solve the task given, we built the solution incrementally. In this
way, we could easily test whether our program worked well or not: If
there were some errors, we would be able to find the error in the small
amount of lines added since last time. It made us also repeat the upload
and debug stage multiple times, which made us more proficient in using
the tools and made it easier to use the more complex parts of the tools
the closer to the task solution we got. 
