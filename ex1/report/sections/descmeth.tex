\section{Description and Methodology}

As described by the task description in the compendium \cite{comp}, a
recommended step-by-step approach to this exercise was as follows:

\begin{enumerate}
    \item Setting up and connecting the STK1000 board.
    \item Downloading the skeleton code and then compiling, uploading
    and debugging this code.
    \item Expanding the skeleton code with code to turn on the LEDs.
    \item Writing code to read the values of the buttons.
    \item Further expanding the program with an interrupt handler.
    \item Modifying the interrupt handler to do the work of reading the
    buttons and changing the LED values accordingly.
\end{enumerate}

This is more or less the approach we went with, with the exception of
writing the interrupt handler before the button handling code.

\subsection{Hardware setup and toolchain usage}

To set up the board, the jumpers were set to their given values and IO
cables were connected to the LED and button pins, all as according to
the compendium. The board was connected to the JTAGICE debugging device
with the appropriate connector, while the JTAGICE in turn was connected
to the computer with a USB cable.

Compiling the skeleton code, and also our own code as the development
progressed, was done using the supplied \emph{makefile} with the
utility \texttt{make}. \texttt{make} then took care of calling the
required programs -- the AS assembler, the LD linker and the
microcontroller programmer utility \texttt{avr32program}.

\subsection{IO setup and LED control}

Our first exercise in the code was to get the LEDs to light up. This
required us to learn the workings of the GPIO model of the STK1000
board.

As instructed by the compendium \cite{comp}, the IO ports are controlled
and accessed by addressing into memory space indicated by the
\texttt{AVR32\_PIOB} and \texttt{AVR32\_PIOC} constants. However, we
quickly realized that these 32-bit addresses would not fit as immediate
values in any instruction for a 32-bit RISC platform.

To accomodate these large addresses, they would need to be stored
together with the program code in the object files and thus in memory
when run, such that they would be accessible at a pre-calculated offset
from the program counter. By labelling the data values in the assembly
code, the compiler would take care of calculating these offsets.

Thus, after having defined the constants with e.g. \texttt{piob: .int
AVR32\_POIB}, they could then be accessed by using the \texttt{lddpc}
instruction, for instance like this: \texttt{lddpc r0, piob}.
\texttt{lddpc} means :wq

\usetikzlibrary{arrows,shapes}

\begin{figure}
\centering
\begin{tikzpicture}[auto,node distance=1cm]

    \node[codelabel] (start) {\texttt{\_start}};
    \node[codepart] (setup-regs) [below of=start] {Register setup};
    \node[codepart] (setup-io) [below of=setup-regs] {I/O setup};
    \node[codepart] (setup-int) [below of=setup-io] {Interrupt setup};
    \node[codelabel] (mainloop) [below of=setup-int] {\texttt{mainloop}};
    \node[codepart] (leds) [below of=mainloop] {Set LEDs};
    \node[guide] (leds-left) [left of=leds] {};
    \node[codepart] (sleep) [below of=leds] {Put CPU to sleep};

    \path[->] (start) edge (setup-regs);
    \path[->] (setup-regs) edge (setup-io);
    \path[->] (setup-io) edge (setup-int);
    \path[->] (setup-int) edge (mainloop);
    \path[->] (mainloop) edge (leds);
    \path[->] (leds) edge (sleep);

    \draw[->]
        (sleep.south)
        -- +(0,-0.25)
        -| (sleep.west)
        |- (mainloop.west);

\end{tikzpicture}
\caption{Main program flow}
\label{fig:mainflow}
\end{figure}

\usetikzlibrary{arrows,shapes}

\begin{figure}
\centering
\begin{tikzpicture}[auto,node distance=1cm]

    \node[codelabel] (handler) {\texttt{interrupt\_handler}};
    \node[codepart] (sleeploop) [below of=handler] {Debouncing sleep loop};
    \node[codepart] (isr) [below of=sleeploop]
        {Read Interrupt State Register};

    \node[codelabel] (left) [below of=isr,xshift=-3cm]
        {\texttt{interrupt\_button\_left}};
    \node[codepart] (left-buttonchk) [below of=left]
        {Check left button};
    \node[codepart] (left-ledchk) [below of=left-buttonchk]
        {Check LED position};
    \node[codepart] (left-shift) [below of=left-ledchk]
        {Shift LED setup register left};

    \node[codelabel] (right) [below of=isr,xshift=3cm]
        {\texttt{interrupt\_button\_right}};
    \node[codepart] (right-buttonchk) [below of=right]
        {Check right button};
    \node[codepart] (right-ledchk) [below of=right-buttonchk]
        {Check LED position};
    \node[codepart] (right-shift) [below of=right-ledchk]
        {Shift LED setup register right};

    \node[codelabel] (end) [below of=isr,yshift=-4cm]
        {\texttt{interrupt\_end}};
    \node[codepart] (rete) [below of=end]
        {Return from interrupt handler (\texttt{rete})};


    \path[->] (handler) edge (sleeploop);
    \path[->] (sleeploop) edge (isr);
    \path[->] (isr) edge (left);
    \path[->] (left) edge (left-buttonchk);
    \path[->] (left-buttonchk) edge (left-ledchk);
    \path[->] (left-ledchk) edge (left-shift);
    \path[->] (left-shift) edge (end);
    \path[->] (isr) edge (right);
    \path[->] (right) edge (right-buttonchk);
    \path[->] (right-buttonchk) edge (right-ledchk);
    \path[->] (right-ledchk) edge (right-shift);
    \path[->] (right-shift) edge (end);
    \path[->] (end) edge (rete);

\end{tikzpicture}
\caption{Program flow in interrupt handler}
\label{fig:interruptflow}
\end{figure}

\begin{table}
\centering
\begin{tabular}{ c | l }
    \hline \hline
    \textsc{Register} & \textsc{Usage} \\ \hline
    \texttt{r0} &       Base address for I/O port B (\texttt{PIOB}). \\
    \texttt{r1} &       Base address for I/O port C (\texttt{PIOC}). \\
    \texttt{r2} &       Base address for interrupt controller
                        (\texttt{INTC}). \\
    \texttt{r8} &       Scratch register 1. \\
    \texttt{r9} &       Scratch register 2. \\
    \texttt{r11} &      Constantly \texttt{0xff} (to save a few
                        instructions). \\
    \texttt{r12} &      The current LED setup (in bits 0-7). \\
    \hline
\end{tabular}
\caption{Overview of register usage}
\label{tab:registers}
\end{table}

