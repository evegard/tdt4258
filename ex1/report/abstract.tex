Many products today need an embedded system in order to work as
intended. They take data from some input, process the data on a
microprocessor, and possibly sends data on some output-port if that is
the desired functionality. This report investigates how the AVR32
microprocessor from Atmel handles I/O and interrupts at assembly level,
as well as how programming such a microprocessor works. This was done by creating a small program in assembly code where the
user is able to control the position of an LED-paddle by pressing
buttons.  We also learnt how and why bouncing occurs and how we can prevent the errors it can cause. The report also explains how to set up and enable interrupts through
autovectors, and explains instruction behaviour which people not
familiar with assembly language may not know about or may forget during
programming.

\begin{comment}
TODO:
- rask konklusjon 
\end{comment}

