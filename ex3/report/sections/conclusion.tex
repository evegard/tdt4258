\section{Conclusion}

This assignment was a good introduction on how to work with Linux, as
well as how to implement and use kernel drivers on a microcontroller. We
started out by implementing a kernel driver for the buttons and LEDs on
the STK1000 board, before we implemented a game through usage of the
previously implemented device, a sound device and a frame buffer
device.

Most of the issues in this assignment were related to hardware
limitations and assumptions about specifications which we later found
out were wrong. For instance, we had to tune the sampling rate on the
files as well as in the sound driver settings so that the sounds played at correct speed, and we had to
bit-shift because the endianness of the TGA-files and the frame buffer
were different.

The assignment provided not only technical knowledge and insight, but
also experience with relatively large C projects where one has to
design an interface for others to use and plan functionality of the code
prior to its implentation. As both are highly relevant for students
studying complex computing systems, we found the assignment very
rewarding.
