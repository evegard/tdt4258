\section{Introduction}

The objective of this task is that the group members will get a better
understanding in how device drivers in Linux work, how to implement a
character device driver in C for the buttons and LEDs on the STK1000
board and experience on how to use device drivers in a relatively large
C program by making a game named {\em Scorched Land}. The task given to
us in the course Microcontroller System Design (TDT4258) to learn these
things was to \\
``{\em Make a driver for the use of buttons and LEDs on the STK1000. It
should be implemented as a kernel module. [\ldots] Complete the game.
Use \texttt{/dev/fb0} directly for writing to LCD screen. Use
\texttt{/dev/dsp} for producing the sound. Use your own driver for
reading the status of the buttons on STK1000. Use also your own driver
for the control of the LED diodes.  These can be used, for example, to
show how many lives a player has left or some other status information
about the game. Or you can just blink in some nice repetitive
way.}''\cite{comp}

As the specifications on how the driver exactly should work were left
unspecified, we decided to let the logic be as simple as possible:
Reading a char from the \texttt{/dev/stkboard} module gives information
about which buttons are pushed, and writing a char to the device will
set the LEDs in the configuration specified. The proper specification is
written in \ldots % TODO.
\begin{comment}
\begin{itemize} % Q: Move to descmeth.
\item Reading one char from the device will give you the bitmask of the
buttons pressed - for instance would 129 (\texttt{1000 0001}) mean that
only button 7 and 0 are pushed. 
\item Writing one char to the device will write the bitmask of the LEDs
in the same way you read the buttons pressed from the device - if you
write the char 66 (\texttt{0100 0010}) to the device, then LED 6 and 1
will be lit while the others will be turned off if they aren't already
off. Notice that there's no way to read the LEDs from the device, as
this was not a specification requirement. To do that in a program, we
would save the current configuration of the LEDs we just wrote.
\end{itemize}
\end{comment}

The game was developed by splitting the work in three: One to work on
integrating the device drivers with the GUI and the game, one to work on
creating and finding sounds and graphics as well as implementing code to
properly load those resouces, and one to work on the game logic. By
designing an API for image, sounds and game logic, the code became
very modular and it became very simple to test and debug each module on
its own.

Apart from the requirement that the game should use the driver we made,
\texttt{/dev/fb0} and \texttt{/dev/dsp}, the game logic was not
specified. The game logic was thus developed from the images in the
compendium\cite{comp} and what we already knew about the military
strategy known as scorched earth: The game designed is a turn-based game
where a soldier tries to get up to a tank, and the tank tries to block
the soldier's way by firing its gun to scorch the earth or to hit the
soldier. If there's no way for the soldier to get up to the tank or the
tank manages to hit the soldier, the tank wins. Otherwise, if the
soldier manages to get up to the tank, the soldier wins. A more detailed
specification is listed in the \ldots

