\section{Introduction}

The objective of this task is that the group members will get a better
understanding in how device drivers in Linux work, how to implement a
character device driver in C for the buttons and LEDs on the STK1000
board and experience on how to use device drivers in a relatively large
C program. The task given to us in the course Microcontroller System
Design (TDT4258) to learn these things was to ``Make a driver for the
use of buttons and LEDs on the STK1000. It should be implemented as a
kernel module. [\ldots] Complete the game. Use /dev/fb0 directly for
writing to LCD screen. Use /dev/dsp for producing the sound. Use your
own driver for reading the status of the buttons on STK1000. Use also
your own driver for the control of the LED diodes. These can be used,
for example, to show how many lives a player has left or some other
status information about the game. Or you can just blink in some nice
repetitive way.''\cite{comp}

As the specifications on how the driver exactly should work were left
unspecified, we decided to let the logic be as simple as possible:
\begin{itemize}
\item Reading one char from the device will give you the bitmask of the
      buttons pressed - for instance would 129 (1000 0001) mean that only
      button 7 and 0 are pushed. 
\item Writing one char to the device will write the bitmask of the LEDs
      in the same way you read the buttons pressed from the device - if
      you write the char 66 (0100 0010) to the device, then LED 6 and 1
      will be lit while the others will be turned off if they aren't
      already off. Notice that there's no way to read the LEDs from the
      device, as this was not a specification listed. To do that in a
      program, we would save the current configuration of the LEDs we
      just wrote.
\end{itemize}

%TODO More about... stuff.
