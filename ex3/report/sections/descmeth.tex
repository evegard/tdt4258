\section{Description and Methodology}

Roughly, the assignment consisted of three parts: \emph{uploading Linux}
to the microcontroller board, \emph{developing a kernel driver} for the
LEDs and buttons, and \emph{developing a game} utilizing these.

Developing the game was the most time-consuming part of the assignment.
This part consisted of several components that had to be developed. In
addition to the game logic and the user interface, the game needed
support for rendering to the screen, playing sounds, importing images,
reading hardware buttons and controlling LEDs.

\subsection{Installing Linux}

To get Linux to run on the microcontroller, a boot loader is needed in
the Flash memory as the first stage of the booting process. To upload
this boot loader, \texttt{avr32-program} was used as in the previous
assignments.

The boot loader will look for a Linux kernel on the board's memory
card. The memory card thus needs to contain a valid file system, as well
as the root file system for the OS to boot into. This was all supplied
in an image file that was ready to be written to the memory card.
Writing the image file to the card was done using \texttt{dd} on the
lab computer.

\subsection{Developing the kernel driver}

\subsection{Developing the game}
