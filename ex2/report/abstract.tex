Handling input and output data is essential in all programs. When
developing programs for operative systems, the operative system will
provide a nice abstraction such that receiving and sending data is made
easy. When there is no operating system, one have to handle reading and
writing such data manually. This report shows how to create a C-program
which runs without an operating system on top of an STK1000 board. The
program is able to generate sound and send it to audio devices connected
through the jack socket on the board. This is done by writing to
memory-mapped I/O owned by the internal audio bitstream digital to
analog converter, which in turn handles output to the jack socket. The
I/O is performed in an interrupt routine which is consistently called by
a clock. We also explain and test out different ways of generating
sounds to the interrupt routine fast enough.
