\section{Results and tests}

Testing of the sound production was performed in successive stages, as
we refined the code. Tests consisted of uploading the code and listening
to the result, using the headphones. One of the group members is a hobby
musician with relative pitch and performed the more advanced listening
tests.

\subsection{Prerequisites for testing:}
\begin{itemize}
\item Code compiles and is uploaded to microcontroller.
\item Headphones in jack socket.
\item For pitch-specific tests, person easily able to distinguish between frequencies (relative pitch).
\item Trusted sound generator for pitch comparison to absolute frequencies (we used Audacity 1.3 Beta).
\item Standard pitch is Concert A, 440Hz.
\end{itemize}

\subsection{Terms and expressions used in tables}
\begin{itemize}
\item ``Reasonably clear note'' denotes a clear tone with only light
noise on top. Some light noise is to be expected due to our hardware,
and is ignored for testing purposes, as there is no way to change this.
\item ``Noise'' refers to significant noise, such as low buzzes, high
beeps, and other noise that goes well beyond the expected light noise.
\item ``Consistent'' means that the pitch of the output and the
pitch of the trusted sample sound are indistinguishable.
\end{itemize}

Our first implementation used a random number generator as input.
Test 1:
\begin{center}
\begin{tabular}{|p{3.6cm}|p{3.6cm}|p{3.6cm}|}
\hline
{\sc Test} & {\sc Expected outcome} & {\sc Observed outcome}\\ \hline
Do we have sound? & Yes. & Yes.\\ \hline
Do we have noise? & Yes, given our random input. & Yes. \\ \hline
Do we have a reasonably clear note? & No. & No. We did not have a clear note at all, as expected.\\ \hline
\end{tabular}
\end{center}
After this, we tried implementing a sine wave directly with a frequency
of 440hz.
Test 2:
\begin{center}
\begin{tabular}{|p{3.6cm}|p{3.6cm}|p{3.6cm}|}
\hline
{\sc Test} & {\sc Expected outcome} & {\sc Observed outcome}\\ \hline
Do we have sound? & Yes. & Yes. \\ \hline
Do we have a reasonably clear note? & Yes. & No. {\em Test failed.} \\ \hline
Do we have noise beyond this? & No. & No. \\ \hline
Do we have a pitch consistent with A440? & Yes. & No. {\em Test failed.} \\
\hline
\end{tabular}
\end{center}

As a result of this test, we moved the sine calculation out of the
interrupt handler, and into a sine lookup table.
Test 3:
\begin{center}
\begin{tabular}{|p{3.6cm}|p{3.6cm}|p{3.6cm}|}
\hline
{\sc Test} & {\sc Expected outcome} & {\sc Observed outcome}\\ \hline
Do we have sound? & Yes. & Yes. \\ \hline
Do we have a reasonably clear note? & Yes. & Yes. \\ \hline
Do we have noise beyond this? & No. & No. \\ \hline
Do we have a pitch consistent with A440? & Yes. & No. {\em Test failed.} \\
\hline
\end{tabular}
\end{center}

However, due to ending up with the wrong tone we suspected that our interrupt handler still took up too much
time, and that samples were not delivered sufficiently fast enough.

We tested this by successively dividing the rate by 2, 4, and 8. Seeing as this was the only difference in our code, we only compared the resulting pitch with 440Hz.
\begin{center}
\begin{tabular}{|l|l|}
\hline
{\sc Rate divisor} & {\sc Consistent with A440} \\ \hline
2 & No \\ \hline
4 & No \\ \hline
8 & Yes \\ \hline
\end{tabular}
\end{center}

We concluded that our interrupt handler took too much time by a factor
of approximately eight. As a result of this, we realised that our method
was still too slow, and decided to change our lookup table entirely,
changing to a sound buffer. The sound buffer contains all the
pre-calculated samples. //TODO:(REF to earlier text!) After this change,
we decided to test it again.

Test 4:
\begin{center}
\begin{tabular}{|p{3.6cm}|p{3.6cm}|p{3.6cm}|}
\hline
{\sc Test} & {\sc Expected outcome} & {\sc Observed outcome}\\ \hline
Do we have sound? & Yes. & Yes. \\ \hline
Do we have a reasonably clear note? & Yes. & Yes. \\ \hline
Do we have noise beyond this? & No. & No. \\ \hline
Do we have a pitch consistent with A440? & Yes. & Yes. \\ \hline
\end{tabular}
\end{center}

We were now able to produce a note given a certain frequency. We
continued by implementing octaves. Octaves are half or double the
frequency, and are perceived by the human ear to be alike. This makes
them easy to test for. We used the buttons to control this. Pressing the
left button (button 3) would make the function double the frequency (up
one octave), and pressing the right button (button 1) would halve the
frequency (going down one octave).

Test 5:
\begin{center}
\begin{tabular}{|p{3.6cm}|p{3.6cm}|p{3.6cm}|}
\hline
{\sc Test} & {\sc Expected outcome} & {\sc Observed outcome}\\ \hline
Do we have sound? & Yes. & Yes. \\ \hline
Do we have a reasonably clear note? & Yes. & Yes. \\ \hline
Do we have noise beyond this? & No. & No. \\ \hline
Do the buttons react to pressing? & Yes. & Yes. \\ \hline
Does the left button shift the pitch up? & Yes. & Yes. \\ \hline
Does the right button shift the pitch down? & Yes. & Yes. \\ \hline
At button press, is the pitch change up or down consistent to octaves? &
Yes. & Yes. \\ \hline
\end{tabular}
\end{center}

We then implemented the chromatic scale using the standard semitone
frequency ratio [Twelfth root of two] and using 440Hz as a reference
point, binding the first eight semitones starting from C4 to the eight
buttons, in theory giving us a range of C4 to G4. If correct, the pitch
would increase by a semitone for each higher button pressed, and the
buttons for C (1), E (5), and G(8) would give us a sequence called a
major triad (in this case a C major triad). This sequence is very easily identifiable even to
non-musicians, as it is found in most popular music.

Test 6:
\begin{center}
\begin{tabular}{|p{3.6cm}|p{3.6cm}|p{3.6cm}|}
\hline
{\sc Test} & {\sc Expected outcome} & {\sc Observed outcome}\\ \hline
Do we have sound? & Yes. & Yes.\\ \hline
Do we have a reasonably clear note? & Yes. & Yes. \\ \hline
Do we have noise beyond this? & No. & No. \\ \hline
Do the buttons react to pressing? & Yes. & Yes.\\ \hline
Does a successive pressing of buttons 1-8 play the correct notes? & Yes. & No.\\ \hline
Does the C major triad play correctly? & Yes. & No.\\ \hline
\end{tabular}
\end{center}

At first, we were puzzled at this, and started checking whether we had made an error in calculating the right frequency. However, double-checking the code showed that we had indeed used the
correct formula for the pitch as far as we could see, and we
decided to test it again by implementing note sequence arrays
(melodies), instead of directly binding the tones to the buttons. For
melodies, we chose simple, well-known children's songs so that the
entire group would easily be able to identify any errors.

Test 7:
\begin{center}
\begin{tabular}{|p{3.6cm}|p{3.6cm}|p{3.6cm}|}
\hline
{\sc Test} & {\sc Expected outcome} & {\sc Observed outcome}\\ \hline
Do we have sound? & Yes. & Yes.\\ \hline
Do we have a reasonably clear note? & Yes. & Yes. \\ \hline
Do we have noise beyond this? & No. & No. \\ \hline
Do the buttons react to pressing? & Yes. & Yes.\\ \hline
Do the melodies perform in tune as coded with pitch references? & Yes. &
Yes. \\ \hline
\end{tabular}
\end{center}

Our test established that the fault must have been somewhere else, and we
continued implementing melodies and note sequences as planned. We
decided that we also needed to find a way to make silences, as our
melodies needed short pauses between certain notes.

\begin{center}
\begin{tabular}{|p{5cm}|p{5cm}|}
\hline
{\sc Attempted solution} & {\sc Result} \\ \hline
Very low frequency.	& Faint crackling noise, buzzing and beeping.  \\ \hline
Very high frequency. & Loud buzz.  \\ \hline
Gradually lowering the amplitude. & Faint crackling noise, buzzing and beeping. Gradual fade-out of tone. \\ \hline
\end{tabular}
\end{center}

We have not been able to eliminate this noise, but eventually ended up
choosing to use a very low frequency, as this noise was the least
annoying. This was implemented by using a buffer with one element of
value 0. Having decided this, we implemented ``Itsy Bitsy Spider'' with
pauses as our introduction melody, along with a few other sequences for
a win, firing, and a hit.

Test 8:
\begin{center}
\begin{tabular}{|p{3.6cm}|p{3.6cm}|p{3.6cm}|}
\hline
{\sc Test} & {\sc Expected outcome} & {\sc Observed outcome}\\ \hline
Do we have sound? & Yes. & Yes. \\ \hline
Do we have a reasonably clear note? & Yes. & Yes. \\ \hline
Do we have noise beyond this? & No. & Yes, we did, during the recently implemented silences. \\ \hline
Reasonably clear note? & Yes. & Yes. \\ \hline
Do the buttons react to pressing? & Yes. & Yes. \\ \hline
Do the melodies perform in tune as coded with pitch references? & Yes. &
Yes. \\ \hline
\end{tabular}
\end{center}
