\section{Description and Methodology}

As recommended in the compendium \cite{comp}, we followed the approach
outlined below in doing this exercise.

\begin{enumerate}
    \item Compiling, uploading and debugging the handout code.
    \item Recreating the previous assignment in C code.
    \item Implementing the current assigment.
\end{enumerate}

The steps required to implement the current assignment were as follows:

\begin{enumerate}
    \item Configuring I/O for LED control.
    \item Configuring I/O for button handling and implementing a button
    interrupt routine.
    \item Setting up the ABDAC (\emph{Audio Bitstream Digital to Analog
    Converter}).
        \begin{enumerate}
            \item Setting up a clock source for the ABDAC.
            \item Configuring the ABDAC.
            \item Implementing an interrupt routine supplying samples to
            the ABDAC.
        \end{enumerate}
    \item Implementing sounds/tones.
    \item Implementing melodies.
    \item Letting melodies to be controlled by the buttons.
\end{enumerate}

\subsection{Implementation Overview}
\subsection{LED and Button Control}
\subsection{Playing Sound Samples}
\subsection{Playing Melodies}

\begin{comment}
Vi burde kommentere litt om math.h, M\_2\_PI-moroa, samt problemer med
lydgenereringen. Også kommentere at sinus er alt for treg til å kunne
brukes direkte. 

Tidligere hadde vi problemer med å regne ut toner og lignende, og vi var
usikre på hvorfor vi fikk spraking o.l. Vi prøvde også å sette ned
raten, og det fungerte av ukjente grunner.  Det viser seg at en
kombinasjon av approksimasjoner av sinus samt mye multiplikasjon gjorde
interrupt-handleren tregere enn det vi trodde - og det å sette ned raten
gjorde det mulig interrupt-handleren å produsere lyden vi ønsket, men
med en dårligere kvalitet.
\end{comment}

