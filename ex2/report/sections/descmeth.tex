\section{Description and Methodology}

As recommended in the compendium \cite{comp}, we followed the approach
outlined below in doing this exercise.

\begin{enumerate}
    \item Compiling, uploading and debugging the handout code.
    \item Recreating the previous assignment in C code.
    \item Implementing the current assigment.
\end{enumerate}

The steps required to implement the current assignment were as follows:

\begin{enumerate}
    \item Configuring I/O for LED control.
    \item Configuring I/O for button handling and implementing a button
    interrupt routine.
    \item Setting up the ABDAC (\emph{Audio Bitstream Digital to Analog
    Converter}).
        \begin{enumerate}
            \item Configuring I/O to allow using the ABDAC.
            \item Setting up a clock source for the ABDAC.
            \item Configuring the ABDAC.
            \item Implementing an interrupt routine supplying samples to
            the ABDAC.
        \end{enumerate}
    \item Implementing sounds/tones.
    \item Implementing melodies.
    \item Letting melodies to be controlled by the buttons.
\end{enumerate}

\subsection{Implementation Overview}

The source code files in our final implementation are listed in table
\ref{tab:sourcefiles}.

\begin{table}[H]
\centering
\begin{tabular}{l|l}
\hline
\textsc{File} & \textsc{Function} \\ \hline
\texttt{ex2.c} & \texttt{main} function and initial setup. \\
\texttt{ex2.h} & Header file for \texttt{ex2.c}. \\
\texttt{button.c} & Button I/O setup and interrupt routine. \\
\texttt{button.h} & Header file for \texttt{button.c}. \\
\texttt{led.c} & LED I/O setup and functions to set and get LED values. \\
\texttt{led.h} & Header file for \texttt{led.c}. \\
\texttt{sound.c} & Sound buffer management, ABDAC setup and interrupt
routine. \\
\texttt{sound.h} & Header file for \texttt{sound.c}. \\
\texttt{melody.c} & Melody playing. \\
\texttt{melody.h} & Header file for \texttt{melody.c}. \\
\hline
\end{tabular}
\caption{Overview of source code files.}
\label{tab:sourcefiles}
\end{table}

The program is spread across these files to separate the code into
logically separate units/modules. The intention was to have separate
modules for button handling, LED control, sound playing and melody
playing.

\subsection{LED and Button Control}

The LED control code implemented in \texttt{led.c} works similarly to
the assembly code in the previous exercise. The same control registers
are accessed and used the same way. In this C implementation, however,
we utilize the AVR header files to access the various flags and fields
as C data structures rather than as bit offsets.

\texttt{led\_init} initializes the I/O pins for LED control. The global
variable \texttt{led\_setup} contains the current LED state.
\texttt{led\_update} updates the data pins to reflect the value in
\texttt{led\_setup}. The utility functions \texttt{led\_set},
\texttt{led\_get} and \texttt{led\_clear} simplify common LED operations
by manipulating \texttt{led\_setup} accordingly and then optionally
calling \texttt{led\_update}.

Button handling is also similar to the assembly code in the previous
exercise. \texttt{btn\_init} initializes the I/O pins for input and
registers \texttt{btn\_interrupt} as the interrupt routine for button
changes. \texttt{btn\_interrupt} is an extended version of the interrupt
routine in the previous exercise, checking the status of \emph{all} of
the eight buttons in a \texttt{for} loop. When a button is pushed,
\texttt{mel\_set\_melody} is called to change the melody.

\subsection{Playing Sound Samples}

\subsubsection{ABDAC Setup}
To generate sound, the ABDAC first needs to be set up. As outlined in
the beginning of the chapter, there are several steps in doing this.

First, the I/O pins shared with the ABDAC module, pins 20 and 21 on port
B, need to be relinquished. This is done by setting the corresponding
bits in PIOB's \texttt{PDR} (PIO Disable Register) and \texttt{ASR}
(Peripheral A Select Register) registers.

Next, the ABDAC needs a clock source to time the samples. This is
supplied by a \emph{Power Manager} on the chip. The Power Manager has
several configurable generic clocks, and the ABDAC is connected to clock
number 6. To configure this clock output, we need to specify an internal
clock source and an optional frequency divisor. In our program, we use
oscillator 0 and no frequency divison. This gives a clock rate of 20
MHz. \cite{ap7000}

Lastly, we need to specify an interrupt routine supplying the samples
(\texttt{snd\_interrupt}), and then the ABDAC needs to be activated.
This is done by setting two flags in the ABDAC registers.

These initialization steps are all contained in the function
\texttt{snd\_init}.

The ABDAC will, after activation, trigger the interrupt routine
\texttt{snd\_interrupt} every time it needs a new sample. The sample
rate of the ABDAC is the rate of the clock source divided by 256. In our
case, the sample rate is $\frac{20000000 Hz}{256}=78125 Hz$.

\subsubsection{Supplying Sound Samples}
To begin with, we only supplied random data for the sample values. This
allowed us to verify a working ABDAC before implementing sounds.
Afterwards, we moved on to implementing sinusoidal sound waves -- sound
waves based on sine values.

Experimentation showed us that evaluating the \texttt{sin} function in
the interrupt routine was too slow. An initial attempt at alleviating
this, was to use a lookup table for the sine values. However, this
optimization was also insufficient. The interrupt routine would still
need to do multiplications and divisions to get the appropriate sample,
and this was too slow.

Our solution is to \emph{precalculate} all of the samples needed for a
cycle/wavelength of the current sound wave. These samples are put in a
\emph{sound buffer}. The interrupt routine only needs to maintain a step
counter, read into the sound buffer correspondingly and return these
values. The interrupt routine is left as a quick buffer lookup, and the
code for generating specific sounds can be put elsewhere.



\subsection{Playing Melodies}
\subsection{Ideas for Improvement}
